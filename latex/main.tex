\documentclass[a4paper,man,natbib]{apa6}

\usepackage[spanish]{babel}
\usepackage[utf8x]{inputenc}
\usepackage{amsmath}
\usepackage{graphicx}
\usepackage[colorinlistoftodos]{todonotes}

\usepackage[hyphens]{url}
\usepackage[hidelinks]{hyperref}
\hypersetup{breaklinks=true}
\urlstyle{same}
%\usepackage{cite}


\shorttitle{Prototipo de aplicación IOT para la gestión del consumo de agua y energía en el hogar.}

\begin{document}

\begin{titlepage}
    \begin{center}
        \vspace*{1cm}

        \begin{Huge}
            \shorttitle{Tarea 5 - Planteamiento del problema}
        \end{Huge}

        \includegraphics[width=0.22\textwidth]{img/universidadDelValle.png}
        \vfill

        Ensayo

        \vfill
        \textbf{Prototipo de aplicación IOT para la gestión del consumo de agua y energía en el hogar.}\\
        \vfill
        
        Estudiantes\\
        Diego Fernando Chaverra Castillo 201940322\\
        Juan Camilo Santa Gomez 201943214\\
        Sebastián Afanador Fontal 201629587\\
        \vfill
        Profesora\\
        Diana Marcela Mendoza. Mgs
        
        
        \vfill

        \textbf{Introducción a la Gestión Ambiental}
        
        \vfill
           
        Facultad de Ingeniería\\
        Escuela de Ingeniería de Recursos Naturales y del Ambiente\\
        Universidad del Valle\\
        Cali - Colombia\\
        \vfill
        14 de Julio del 2022

    \end{center}
\end{titlepage}

\section{Introducción.}
En el presente ensayo se aborda como tema la creación de un prototipo de aplicación IOT para la gestión del consumo de agua y energía en el hogar. Se plantea como punto de partida elaborar una estrategia de información comunicación y educación ambiental en el hogar para la mejora del consumo de energía y agua a través de una aplicación de IOT; para desarrollar este propósito se estableció como primer punto informar los datos de consumo de agua y energía en un dashboard y a partir del envío de correo electrónico para la muestra de información en tiempo real.; como segundo punto socializar las piezas de comunicación de la aplicación a partir de gráficas, promedios, registros comparativos e históricos de consumo para la educación y el desarrollo de una noción cuantitativa de consumo; y finalmente como tercer punto analizar la información de consumo y la experiencia en el hogar para la identificación de oportunidades de mejora. Acorde a ello se aborda la cuestión sobre ¿Cómo se puede elaborar una estrategia de comunicación, información y educación de gestión ambiental en el hogar para la mejora del consumo de agua y energía usando la tecnología qué sea accesible a la mayor cantidad de hogares posibles? De esta manera, la estrategia metodológica estuvo orientada a la revisión bibliográfica de artículos científicos y noticias entre las qué se encuentran los autores \citep{7398710}, \citep{8612412}, \citep{Aguaenel98:online}, \citep{Ahorrode59:online}, \citep{Betancourt20201}, \citep{Diadeagu32:online}, \citep{Enterate82:online}, \citep{Milesdem41:online} que permitieron un acercamiento a la discusión sobre cómo usar el Internet de las cosas y el desarrollo de software en la gestión ambiental en el hogar. Finalmente, se destaca una reflexión final que permiten concluir que el presente ejecicio permitió la apropiación de diversos saberes y la oportunidad de llevar a la práctica parte de la discusión desarrollada en este ensayo.\newline

De esta manera se puede encontrar para justificación para este ejercicio que dentro de muchos hogares no hay un referente objetivo qué ayude a mejorar el consumo del agua y la energía, en la mayoría de los casos no tenemos una forma de contrastar nuestros hábitos con medidas claras qué nos retroalimentan en la gestión qué le hacemos a estos recursos.\newline

Actualmente la tecnología es una herramienta accesible para muchas personas en la qué encontramos un punto de apalancamiento para desarrollar un canal de información de dos vías entre los procesos de consumo y su regulación, según el periódico El País:\newline

\begin{quote}
    \vspace{5pt}
    {\small
        En el hogar y aunque apliquemos el sentido común, hay todavía mucho trabajo por hacer para lograr un consumo eficiente del agua y por fortuna, contamos con la asistencia de la tecnología que a través de aplicaciones y dispositivos, harán que este ahorro sea además, cómodo.
    }
    \begin{flushright}
        (El Pais, 2016, Párr. 2)
    \end{flushright}

\end{quote}

Construir una aplicación para educar a las personas y obtener información en tiempo real del consumo de agua y energía del qué participa cada miembro del hogar es un factor estratégico para promover un hábito de consumo responsable aplicando estrategias de gestión ambiental en el hogar.\newline

El primer paso para enfrentar un problema es identificarlo, si no se conoce el enemigo entonces no podrá ser vencido. En términos de gestión ambiental en el hogar los problemas de abordan con una filosofía de “la rana en la olla”,  la temperatura de agua de la olla va subiendo mientras qué la rana disfruta en la olla el agua caliente, poco a poco la temperatura incrementa hasta qué la rana olvida qué la temperatura está subiendo tanto qué peligra su propia vida, relajada y ajena al problema la rana permanece hasta cuando es demasiado tarde para salir de la olla porque la rana ha muerto. Cuando en un hogar el consumo de recursos es alto o simplemente podría mejorarse quizá a nadie le interesa tanto como a la persona qué debe pagar la cuenta a final de mes.\newline

A pesar qué la tierra tiene ¾ partes de su superficie cubiertas por agua solamente el 2,5\% de esta es agua dulce y sólo el 0,007\% es apta para el consumo humano \citep{Aguaenel98:online}. En la mayoría de los casos el agua qué sale por las llaves de una residencia debe ser tratada para ser apta para el consumo, es por esta razón qué el agua potable merece la atención de la humanidad, su disponibilidad a nivel mundial es aproximadamente del 75\%, \citep{Milesdem41:online} esto quiere decir qué 1 de cada 4 personas en el mundo no cuenta con acceso a agua potable. Por estas razones, la misma existencia de la civilización humana y por justa causa se debe la entera atención a garantizar su disponibilidad para el consumo y en otros casos para la generación de energía dignificando y razonando su consumo al necesario.\newline

La tecnología es una herramienta poderosa para transmitir y comunicar información clave para la educación. En Colombia el costo de vida no permite a los hogares implementar soluciones tecnológicas qué ayuden a la mejora de la calidad de vida y la relación con el medio ambiente, como por ejemplo los sistemas de ahorro energético basados en el aprovechamiento de la energía solar o los sistemas para el aprovechamiento de las aguas lluvias, mucho menos los sistemas de ahorro inteligente qué proponen empresas como Neu. En muchos casos los temas alrededor de este problema son ajenos y de poca importancia para las personas, existe una razón simple para esto y es porque lo ignoran casi completamente. ¿Cómo se puede elaborar una estrategia de comunicación, información y educación de gestión ambiental en el hogar para la mejora del consumo de agua y energía usando la tecnología qué sea accesible a la mayor cantidad de hogares posibles ?.\newline

\section{Discusión.}

\subsection{Aspectos técnicos.}
Para desarrollar este propósito se estableció como primer punto informar los datos de consumo de agua y energía en un dashboard y a partir del envío de correo electrónico para la muestra de información en tiempo real.\newline

\subsubsection{Programación reactiva.}
El paradigma de programación web reactiva ayuda a la creación rápida de aplicaciones responsivas.
Librerías como React facilitan la escritura de código de manera rápida y práctica, Este tipo de tecnologías están al alcance de cualquier persona qué las necesite, solo basta con dedicar algunas horas para leer la documentación del desarrollador de las librerías o quizás si es el caso de una persona qué se está iniciando en la programación, definir algunos conceptos básicos previos al desarrollo de aplicaciones.
\newline

El uso de tecnologías actuales facilita el mantenimiento y funcionamiento de una aplicación. Cuando se trata de crear nuevas características para una aplicación, es necesario tener las herramientas qué facilitan al programador escribir código de manera eficaz y práctica. Dado el caso de qué la aplicación se puede desactualizar, sea mejorada después de la primera versión entonces será fácilmente incluir las nuevas librerías libres de fallos de seguridad y problemas técnicos.
\newline

\subsubsection{Sockets.}
La comunicación de los dispositivos embebidos se facilita cuando se hace en tiempo real. Desde una llamada con un ser querido en una ubicación geográfica diferente a la suya hasta la forma en qué se comunican las principales aplicaciones en tiempo real como los chats y demás aplicaciones de mensajería instantánea.
\newline

La información en tiempo real es muy valiosa para crear una noción fundamentada. Los sockets ayudan a tener esta información disponible para ser transmitida sobre Internet a una gran cantidad de usuarios de una aplicación; En el caso de aplicación explícito qué tiene en el proyecto, estos ayudarán a conocer el consumo en tiempo real de las variables de agua y energía.\newline

\subsubsection{Raspberry Pi y sensores.}
El IOT es una herramienta importante para el reconocimiento de patrones. Desde qué las computadoras se han vuelto tan accesibles desde el punto de vista económico y técnico, tener una computadora qué cabe en la palma de la mano ha permitido ubicar en casi cualquier lugar donde haya conexión a Internet y un punto de energía, todo esto con el fin de recolectar información de nuestros hábitos de consumo (en nuestro caso en particular) y las miles de posibilidades de medir un sin número de variables a partir de la información qué puede ser capturada en un sensor conectado a un conversor de señal digital o una entrada analógica.
\newline

La información recopilada por un sensor es una mirada objetiva de la realidad. No es cuando podemos ver el consumo qué tenemos qué tenemos conocimiento qué nuestros hábitos de consumo y además el uso inadecuado de algunos electrodomésticos puede tener una repercusión notable en lo qué podemos llegar a consumir en un mes, o quizá en toda una vida.\newline

\subsubsection{Correo electrónico.}
La forma más económica y efectiva de llegar a notificar información sin gastar papel es el correo electrónico. Hoy en día es muy común ver cómo el usuario promedio tiene la oportunidad de acceder al correo electrónico y a los mensajes qué ahí quedan registrados. Junto con las tecnologías nombradas anteriormente, el correo electrónico es una de las más útiles para llevar información en cualquier momento y lugar y a cualquier persona sin costo alguno.

\subsection{Uso de la aplicación.}
Como segundo punto socializar las piezas de comunicación de la aplicación a partir de gráficas, promedios, registros comparativos e históricos de consumo para la educación y el desarrollo de una noción cuantitativa de consumo.\newline

\subsubsection{Scrum Dailys.}
De la monitorización constante de un proyecto depende su éxito. Generalmente al final de cada día cuando las personas qué comparten un mismo techo están sentadas compartiendo los alimentos, puede hacerse una labor de socialización de la información qué brinda la aplicación con respecto a las labores cotidianas, la parte más complicada viene después cuando se busca qué labores diarias son objeto de optimización replanteamiento para evitar el alto consumo de energía y agua, Siempre qué haya una buena comunicación y se busquen alternativas viables, es posible mejorar los hábitos de consumo.
\newline

Basarse en las metodologías existentes permite apalancar el proceso para encontrar una solución. En el desarrollo de software se debe coordinar la puesta en marcha de una aplicación mientras qué varias personas deben trabajar aportando su conocimiento y experiencia al mismo tiempo, mencionando qué se está haciendo para mejorar hoy y qué podría llegarse a hacer mañana; es de esta manera como puede llegarse a un consenso y se pueden hacer mejoras qué realmente representen un cambio significativo.\newline

\subsubsection{Cálculo de consumo.}
La potencia eléctrica es una unidad de medida qué involucra el trabajo y el tiempo. En los recibos de servicios públicos se usa el Kilovatio Hora como la unidad de medida de consumo eléctrico, esta magnitud física en el sector doméstico representa el cálculo de la potencia eléctrica de mil watts qué es entregada en una hora por la empresa de servicios públicos. la potencia a su vez está definida como la capacidad de hacer un trabajo en el menor tiempo posible. Al circular por un cable los electrones qué pasan generan un pequeño campo eléctrico qué puede ser detectado por un sensor qué convierte este flujo en una salida analógica con un valor entre 0 y 1V en relación a 0 y 50A de flujo de corriente por el cable.
\newline

Cuando se habla de consumo de agua la medida estandarizada es el metro cúbico. La razón de esto es porque en una residencia de manera quincenal o mensual se consumen tantos litros de agua qué no es práctico medirlos de esta manera sino usando un factor de $1x10^3$ litros (un metro cúbico), los sensores qué miden el flujo de agua actúan simplemente con un mecanismo de giro a razón del paso de agua en una tubería, a medida qué haya mas paso el mecanismo gira más rápido y la cantidad de giros en un intérvalo de tiempo será testigo de la cantidad exacta de agua consumida.


\subsection{Estrategias para la educación.}
Finalmente como tercer punto analizar la información de consumo y la experiencia en el hogar para la identificación de oportunidades de mejora.
\newline
\subsubsection{Comparaciones cotidianas.}
El desarrollo de una noción cuantitativa de consumo es una herramienta clave para optimizar un recurso. Una de las mejoras qué debería tener la aplicación en versiones posteriores sería incluir gráfica qué permitan conocer el ahorro o el exceso en el consumo de agua y energía, todo esto especificado por horas de día y días de la semana para hacer seguimiento de una manera más precisa al consumo relacionándolo con momento exactos de día.
\newline

Lo barato sale caro. Y es precisamente cuando se habla de algunos electrodomésticos qué en el corto plazo representan una inversión moderada como los calentadores eléctricos en la ducha, las freidoras de aire y otros electrodomésticos qué funcionan a base de resistencias. Proyectando el consumo en el tiempo todas las formas de electrodomésticos qué generan calor a partir de fuentes de energía como el gas natural permiten tener controlado el consumo eléctrico pero seguramente representan un costo de instalación mucho mayor qué sus contrapartes de funcionamiento eléctrico.\newline

\subsubsection{Proyección del consumo con la información recopilada.}
Conocer el efecto qué tendrá un comportamiento en el presente ayudará a pensar las cosas dos veces. Cuando se usan electrodomésticos qué tienen un alto consumo eléctrico inmediatamente puede verse el comportamiento ascendente de consumo, es muy util ademas conocer el costo qué ha representado el consumo de energía y agua de las actividades hogareñas porque permite el valor qué tendrá la factura de servicios públicos a final de mes y además permite tomar medidas correctivas en tiempo real sobre los patrones de consumo.
\newline

Las mejoras qué son persistentes en el tiempo son las qué tienen efectos significativos. Es muy curioso como las neveras tienden a elevar su consumo normal a razón de la obstaculización de flujo de aire por las unidades qué se encarga de condensar el líquido refrigerante, esto se traduce en un incremento del consumo de energía solamente por falta de mantenimiento; de manera similar las lavadoras acumulan residuos en forma de grasa en los tambores qué hacen qué el agua y el jabón empleados sean cada vez menos eficientes y se necesite más de estos para qué la ropa quede limpia.\newline


\section{Resultados y conclusiones}
\begin{itemize}
    \item En el ejercicio de implementación de los sensores fue bastante complicado hacer qué la información de la medición se conservara segura, en repetidas ocasiones la memoria principal de la computadora quedaba ilegible porque esta era de mala calidad o porque el sensor encargado de medir la tensión eléctrica generaba cargas espurias dentro de los circuitos. Por tal razón siempre qué se quiera conocer esta información es necesario usar instrumentos de calidad e implementaciones qué alguien haya probado alguna vez.

    \item Cuando se trata de encontrar mejoras en el consumo de energía los pequeños cambios prolongados en el tiempo permiten tener efectos notables, tales como el mantenimiento de los electrodomésticos, la regulación de la temperatura de manera natural, la programación de encendido y apagado de servidores y computadores (siempre qué sea posible).

    \item El uso de grifos ahorradores de agua permite controlar el flujo de salida y puede representar un cambio notable a largo plazo en el consumo, sobre todo en las salidas de agua de alto flujo como las duchas o las llaves de los lavaderos.

    \item Conocer información dinámica de los consumos de agua y energía permite generar un proceso de retroalimentación entre las personas y sus hábitos de consumo qué beneficia el uso racionalizado de estos recursos.
\end{itemize}

%usada para mostrar todas las fuentes
\nocite{*}

\Urlmuskip=0mu plus 1mu\relax
\bibliographystyle{plain}
\bibliography{references}


\end{document}