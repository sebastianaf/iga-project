\documentclass[a4paper,man,natbib]{apa6}

\usepackage[spanish]{babel}
\usepackage[utf8x]{inputenc}
\usepackage{amsmath}
\usepackage{graphicx}
\usepackage[colorinlistoftodos]{todonotes}

\usepackage[hyphens]{url}
\usepackage[hidelinks]{hyperref}
\hypersetup{breaklinks=true}
\urlstyle{same}
%\usepackage{cite}


\shorttitle{Prototipo de aplicación IOT para la gestión del consumo de agua y energía en el hogar.}

\begin{document}

\begin{titlepage}
    \begin{center}
        \vspace*{1cm}

        \begin{Huge}
            \shorttitle{Tarea 5 - Planteamiento del problema}
        \end{Huge}

        \includegraphics[width=0.22\textwidth]{img/universidadDelValle.png}
        \vfill

        Ensayo

        \vfill
        \textbf{Prototipo de aplicación IOT para la gestión del consumo de agua y energía en el hogar.}\\
        \vfill
        
        Estudiantes\\
        Diego Fernando Chaverra Castillo 201940322\\
        Juan Camilo Santa Gomez 201943214\\
        Sebastián Afanador Fontal 201629587\\
        \vfill
        Profesora\\
        Diana Marcela Mendoza. Mgs
        
        
        \vfill

        \textbf{Introducción a la Gestión Ambiental}
        
        \vfill
           
        Facultad de Ingeniería\\
        Escuela de Ingeniería de Recursos Naturales y del Ambiente\\
        Universidad del Valle\\
        Cali - Colombia\\
        \vfill
        14 de Julio del 2022

    \end{center}
\end{titlepage}

\section{Introducción}
En el presente ensayo se aborda como tema la creación de un prototipo de aplicación IOT para la gestión del consumo de agua y energía en el hogar. Se plantea como punto de partida elaborar una estrategia de información comunicación y educación ambiental en el hogar para la mejora del consumo de energía y agua a través de una aplicación de IOT; para desarrollar este propósito se estableció como primer punto informar los datos de consumo de agua y energía en un dashboard y a partir del envío de correo electrónico para la muestra de información en tiempo real.; como segundo punto socializar las piezas de comunicación de la aplicación a partir de gráficas, promedios, registros comparativos e históricos de consumo para la educación y el desarrollo de una noción cuantitativa de consumo; y finalmente como tercer punto analizar la información de consumo y la experiencia en el hogar para la identificación de oportunidades de mejora. Acorde a ello se aborda la cuestión sobre ¿Cómo se puede elaborar una estrategia de comunicación, información y educación de gestión ambiental en el hogar para la mejora del consumo de agua y energía usando la tecnología qué sea accesible a la mayor cantidad de hogares posibles? De esta manera, la estrategia metodológica estuvo orientada a la revisión bibliográfica de artículos científicos y noticias entre las qué se encuentran los autores \citep{7398710}, \citep{8612412}, \citep{Aguaenel98:online}, \citep{Ahorrode59:online}, \citep{Betancourt20201}, \citep{Diadeagu32:online}, \citep{Enterate82:online}, \citep{Milesdem41:online} que permitieron un acercamiento a la discusión sobre cómo usar el Internet de las cosas y el desarrollo de software en la gestión ambiental en el hogar. Finalmente, se destaca una reflexión final que permiten concluir que (…).\newline

De esta manera se puede encontrar para justificación para este ejercicio que dentro de muchos hogares no hay un referente objetivo qué ayude a mejorar el consumo del agua y la energía, en la mayoría de los casos no tenemos una forma de contrastar nuestros hábitos con medidas claras qué nos retroalimentan en la gestión qué le hacemos a estos recursos.\newline

Actualmente la tecnología es una herramienta accesible para muchas personas en la qué encontramos un punto de apalancamiento para desarrollar un canal de información de dos vías entre los procesos de consumo y su regulación, según el periódico El PAÍS:\newline

\begin{quote}
    \vspace{5pt}
    {\small
    En el hogar y aunque apliquemos el sentido común, hay todavía mucho trabajo por hacer para lograr un consumo eficiente del agua y por fortuna, contamos con la asistencia de la tecnología que a través de aplicaciones y dispositivos, harán que este ahorro sea además, cómodo.
    }
    \begin{flushright}
        El Pais (2016): 1
    \end{flushright}

\end{quote}

Construir una aplicación para educar a las personas y obtener información en tiempo real del consumo de agua y energía del qué participa cada miembro del hogar es un factor estratégico para promover un hábito de consumo responsable aplicando estrategias de gestión ambiental en el hogar.\newline

El primer paso para enfrentar un problema es identificarlo, si no se conoce el enemigo entonces no podrá ser vencido. En términos de gestión ambiental en el hogar los problemas de abordan con una filosofía de “la rana en la olla”,  la temperatura de agua de la olla va subiendo mientras qué la rana disfruta en la olla el agua caliente, poco a poco la temperatura incrementa hasta qué la rana olvida qué la temperatura está subiendo tanto qué peligra su propia vida, relajada y ajena al problema la rana permanece hasta cuando es demasiado tarde para salir de la olla porque la rana ha muerto. Cuando en un hogar el consumo de recursos es alto o simplemente podría mejorarse quizá a nadie le interesa tanto como a la persona qué debe pagar la cuenta a final de mes.\newline

A pesar qué la tierra tiene ¾ partes de su superficie cubiertas por agua solamente el 2,5\% de esta es agua dulce y sólo el 0,007\% es apta para el consumo humano \citep{Aguaenel98:online}. En la mayoría de los casos el agua qué sale por las llaves de una residencia debe ser tratada para ser apta para el consumo, es por esta razón qué el agua potable merece la atención de la humanidad, su disponibilidad a nivel mundial es aproximadamente del 75\%, \citep{Milesdem41:online} esto quiere decir qué 1 de cada 4 personas en el mundo no cuenta con acceso a agua potable. Por estas razones, la misma existencia de la civilización humana y por justa causa se debe la entera atención a garantizar su disponibilidad para el consumo y en otros casos para la generación de energía dignificando y razonando su consumo al necesario.\newline

La tecnología es una herramienta poderosa para transmitir y comunicar información clave para la educación. En Colombia el costo de vida no permite a los hogares implementar soluciones tecnológicas qué ayuden a la mejora de la calidad de vida y la relación con el medio ambiente, como por ejemplo los sistemas de ahorro energético basados en el aprovechamiento de la energía solar o los sistemas para el aprovechamiento de las aguas lluvias, mucho menos los sistemas de ahorro inteligente qué proponen empresas como Neu. En muchos casos los temas alrededor de este problema son ajenos y de poca importancia para las personas, existe una razón simple para esto y es porque lo ignoran casi completamente. ¿Cómo se puede elaborar una estrategia de comunicación, información y educación de gestión ambiental en el hogar para la mejora del consumo de agua y energía usando la tecnología qué sea accesible a la mayor cantidad de hogares posibles ?.\newline

%usada para mostrar todas las fuentes
\nocite{*}

\Urlmuskip=0mu plus 1mu\relax
\bibliographystyle{plain}
\bibliography{references}


\end{document}