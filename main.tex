\documentclass[a4paper,man,natbib]{apa6}

\usepackage[spanish]{babel}
\usepackage[utf8x]{inputenc}
\usepackage{amsmath}
\usepackage{graphicx}
\usepackage[colorinlistoftodos]{todonotes}

\usepackage[hyphens]{url}
\usepackage[hidelinks]{hyperref}
\hypersetup{breaklinks=true}
\urlstyle{same}
\usepackage{cite}


\shorttitle{Tarea 4 - Entrega bibliográfica}

\begin{document}

\begin{titlepage}
    \begin{center}
        \vspace*{1cm}

        \begin{Huge}
            \shorttitle{Tarea 5 - Planteamiento del problema}
        \end{Huge}

        \includegraphics[width=0.22\textwidth]{img/universidadDelValle.png}
        \vfill

        Ensayo

        \vfill
        \textbf{Prototipo de aplicación IOT para la gestión del consumo de agua y energía en el hogar.}\\
        \vfill
        
        Estudiantes\\
        Diego Fernando Chaverra Castillo 201940322\\
        Juan Camilo Santa Gomez 201943214\\
        Sebastián Afanador Fontal 201629587\\
        \vfill
        Profesora\\
        Diana Marcela Mendoza. Mgs
        
        
        \vfill

        \textbf{Introducción a la Gestión Ambiental}
        
        \vfill
           
        Facultad de Ingeniería\\
        Escuela de Ingeniería de Recursos Naturales y del Ambiente\\
        Universidad del Valle\\
        Cali - Colombia\\
        \vfill
        14 de Julio del 2022

    \end{center}
\end{titlepage}

\section{Título}
Prototipo de aplicación IOT para la gestión del consumo de agua y energía en el hogar.

\section{Objetivos}
\subsection{Objetivo general}
Elaborar una estrategia de información comunicación y educación ambiental en el hogar para la mejora del consumo de energía y agua a través de una aplicación de IOT.

\subsection{Objetivos específicos}
\begin{itemize}
      \item Informar los datos de consumo de agua y energía en un dashboard y a partir del envío de correo electrónico para la muestra de información en tiempo real.
      \item Socializar las piezas de comunicación de la aplicación a partir de gráficas, promedios, registros comparativos e históricos de consumo para la educación y el desarrollo de una noción cuantitativa de consumo.
      \item Analizar la información de consumo y la experiencia en el hogar para la identificación de oportunidades de mejora.
\end{itemize}

\subsection{Justificación}
Dentro de muchos hogares no hay un referente objetivo qué ayude a mejorar el consumo del agua y la energía, en la mayoría de los casos no tenemos una forma de contrastar nuestros hábitos con medidas claras qué nos retroalimentan en la gestión qué le hacemos a estos recursos.\\

Actualmente la tecnología es una herramienta accesible para muchas personas en la qué encontramos un punto de apalancamiento para desarrollar un canal de información de dos vías entre los procesos de consumo y su regulación, según el periódico El PAÍS:\\

\begin{quote}
    \vspace{5pt}
    {\small
    En el hogar y aunque apliquemos el sentido común, hay todavía mucho trabajo por hacer para lograr un consumo eficiente del agua y por fortuna, contamos con la asistencia de la tecnología que a través de aplicaciones y dispositivos, harán que este ahorro sea además, cómodo.
    }
    \begin{flushright}
        El Pais (2016): 1
    \end{flushright}

\end{quote}

Construir una aplicación para educar a las personas y obtener información en tiempo real del consumo de agua y energía del qué participa cada miembro del hogar es un factor estratégico para promover un hábito de consumo responsable aplicando estrategias de gestión ambiental en el hogar.

%usada para mostrar todas las fuentes
\nocite{*}

\Urlmuskip=0mu plus 1mu\relax
\bibliographystyle{plain}
\bibliography{references}


\end{document}

%
% Please see the package documentation for more information
% on the APA6 document class:
%
% http://www.ctan.org/pkg/apa6
%